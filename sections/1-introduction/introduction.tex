\section{Introduction}

Depuis 10 ans, le nombre d'interventions de secours à personne augmente de 8 à 10\% chaque année. Pour continuer à répondre aux demandes en temps acceptable, les services de secours ont migré, il y a quelques années, vers des systèmes informatisés de gestion des interventions, des équipements et du personnel. Cela a permis à chaque agent de pouvoir gérer sa disponibilité pour pouvoir partir en intervention à la demande, de commander ou échanger du matériel, de se former et de communiquer en temps réel. Cependant, ces systèmes sont anciens et l’ensemble forment un assemblage de briques bien séparées qui rend leurs utilisations complexes et peu efficaces. \\

C’est pourquoi, en collaboration avec la caserne des Sapeurs-Pompiers de Saint-Georges-D’espéranche (dans le nord-isère), nous allons réaliser un projet qui répond à des besoins spécifiques de centralisation et de gestions des agents et équipements, sans remplacer l'existant, tout en accentuant la sécurité des systèmes. \\

Le projet se découpe en quatre grandes parties :
\begin{itemize}
  \item \textbf{Gestion du personnel :} obtenir et maintenir à jour automatiquement toutes les informations sur le personnel (nom, grade, visites médicales, formations réalisées, nombre d'interventions, suivi des formations, équipements...) et communiquer avec la caserne (annonces écrites, messages programmés...)\\
  \item \textbf{Gestion du matériel et véhicules :} obtenir et maintenir à jour automatiquement les réserves matérielles (expiration des consommables, suivi de la consommation, commandes automatiques, prévisions...) et des véhicules (nombre d'interventions, fiche d'inventaire modifiable, signalement de pannes/problèmes, rappel contrôle technique...)\\
  \item \textbf{Formation et regroupement des ressources :} centraliser l'ensemble des documents de formations et proposer un support interactif d’entraînement, d'apprentissage et de formations (recherche de référentiels, de techniques, préparation aux formations, cours en ligne avec évaluations indicatives, mises en situation, QCMs...)\\
  \item \textbf{Adaptation et sécurité : } se baser sur les technologies existantes utilisées par le SDIS\footnote{Service Départemental d'Incendie et de Secours : service publique en charge des sapeurs-pompiers} tout en assurant de bout en bout le secret professionnel, la confidentialité et l'intégrité de chaque informations (stockage et communication entre clients-serveurs sécurisés).
\end{itemize}