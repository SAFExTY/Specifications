\section{Introduction}
%%Posséder une introduction (1 page minimum) qui résume les points essentiels du cahier des charges et de donner une vue d'ensemble au jury. Elle doit faire ressortir l'intérêt du projet et mettre en valeur le but final
    Dans le cadre du projet du semestre 2, quatre étudiants se sont réunis pour réaliser un jeu qualitatif qui consiste en la gestion d'une ville axée principalement sur le pôle santé. La volonté initiale était de réaliser un logiciel utile pour une caserne de pompiers mais face aux multiples contraintes du cadre et du projet, ils ont dû se replier vers un projet qui permettrait de sensibiliser les joueurs aux thématiques de santé et de gestion des services de secours. Ce cahier des charges vise à développer notre projet, à la fois sur le plan technique et organisationnel.\\ 
    
    Notre projet est donc un jeu en vue 2D isométrique (voir annexes) dans lequel le joueur doit bâtir sa ville ainsi que le pôle de secours associé. Des incidents (accident de voiture, malaise, maladie...) apparaissent de manière aléatoire et décime sa population. Le but est de sauver un maximum de personnes afin d'accumuler des points nécessaires à la construction de structures adaptées pour prendre en charge les habitants.
