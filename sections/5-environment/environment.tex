% Tout ce qui se rapporte au projet sans en faire partie
\section{Autour du projet}

Parce que le projet ne consiste pas seulement au développement du jeu en lui-même, il faut également les infrastructures qui vont permettre de faire en sorte que le jeu soit plus qu'un projet étudiant.

\subsection{Langages \& outils}
Pour développer le jeu, nous avons choisi d'utiliser le langage C\# que nous apprenons grâce au TP. Cela nous permet de pouvoir manier un langage plutôt impératif ce qui simplifiera en partie la complexité du code. Également, nous avons choisi d'utiliser le moteur de jeu Godot Engine plutôt à la mode en ce moment. Il permet en outre se pouvoir exporter le jeu sur différentes plateformes. De plus, sa licence MIT fait que nous avons pleins pouvoir sur le code tant que nous citons les copyrights. Nous utiliserons principalement Jetbrains Rider et Microsoft Visual Studio avec le langage C\# pour développer le jeu et son contenu.

\subsection{Le site web}
Le jeu doit posséder un site internet présentant et montrant l'avancée du projet, ainsi qu'un espace de connexion pour se créer et modifier les informations de son compte en accord avec la réglementation en vigueur (RGPD). De ce fait, nous prenons en compte qu'il nous faudra dédier du temps à ce site afin que la présentation de notre projet puisse être la plus agréable possible et qu'elle donne envie de jouer au jeu. Nous pensons séparer le site internet en deux parties : une statique avec la présentation du jeu faites avec Bulma Io, un Framework CSS et une partie dynamique qui sera l'espace de connexion des utilisateurs qui sera développé avec en front-end React.js et son serveur web en C\#.